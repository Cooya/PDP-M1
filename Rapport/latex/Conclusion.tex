\chapter*{Bilan}
\addcontentsline{toc}{chapter}{Bilan}

Ce projet nous a permis de nous donner une première expérience du monde du génie logiciel, c'est-à-dire dans l'organisation d'un développement à moyenne envergure, dans la mise en place d'un environnement de travail adapté et de nous permettre d'apprendre à travailler en groupe afin d'amener de tels projets à terme. Bien qu'à la remise du présent rapport le prototype ne soit pas complet, les modules peuvent être réutilisables et maintenables indépendamment des autres, ce qui nous permettrait d'intégrer le module de synchronisation \textit{Bluetooth} à une future application qui pourrait viser les usagers malvoyants grâce à une fonction de vibreur. \\
\\
L'apprentissage de l'univers \textit{Android} a été nécessaire avant de commencer le développement, nous avons donc appris comment fonctionnait un téléphone au niveau du noyau Linux et des applications.
Nous avons aussi découvert un nouvel environnement de travail adapté à \textit{Android} et, bien que l'installation et la mise en place de cet environnement ait été compliqué de prime abord à cause de tous les outils de développement organisationnels qui interagissent entre eux, le développement a été très fluide. Compte tenu de la prépondérance d'\textit{Android} sur les appareils mobiles, notre première expérience avec la plate-forme dédié nous sera très utile dans notre carrière professionnelle. \\
\\
Au final, ce projet ajoute une nouvelle compétence à notre profil, nous permettant si on le souhaite de la mettre en valeur lors d'entretien afin de commencer une carrière dans les applications mobiles et, quand bien même ce choix de carrière serait mis de côté, l'entente, le travail en groupe et l'esprit d'équipe sont des éléments que nous avons compris primordiaux dans le bon développement logiciel.