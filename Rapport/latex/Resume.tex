\begin{abstract}
Ce projet a été réalisé sous la direction de Matthieu Raffinot, chercheur CNRS, dans le cadre du Master 1 Génie logiciel lié au cours de Projet de Programmation, dirigé par Philippe Narbel, maître de conférence. Le projet a pour but pédagogique de nous initier au génie Logiciel, et plus précisément à la gestion et l'organisation d'un projet en équipe.\\

Le sujet proposé par le client est de permettre à une personne malvoyante de pouvoir utiliser une application \textit{Android} de guidage par GPS afin de se rendre à une destination, de préférence à pied. L'application doit donc s'adapter à la vue défaillante de l'utilisateur et ainsi proposer une interface ergonomique. L'utilisateur aura la possibilité d'utiliser deux modes d'utilisation. Le premier mode sera utilisable avec deux téléphones qui symboliseront chacun une direction, le téléphone de gauche vibrera lorsque l'utilisateur devra tourner à gauche et le téléphone de droite vibrera lorsque l'utilisateur devra tourner à droite. Le second mode sera proposé si l'utilisateur ne possède qu'un seul téléphone fonctionnel qui vibrera une fois pour la direction de gauche, et deux fois pour la direction de droite.\\

Pour ce faire, on a donc eu besoin d'exploiter les technologies natives du téléphone : le GPS, les capteurs de navigation inertielle, le vibreur et le \textit{BlueTooth}. L'application sera développée sur un téléphone mobile sous système \textit{Android}, en utilisant le langage de programmation Java, le kit de développement \textit{Android} et l'API \textit{Google} pour la navigation.\\

Il est à noter que l'utilisation de cette application ne remplace pas les moyens habituelles que l'utilisateur utilise pour se déplacer seul (canne, chien d'aveugle, ...). A l'instar d'un GPS classique qui n'exempte pas les conducteurs du respect du code de la route, l'application est uniquement un guide.
\end{abstract}