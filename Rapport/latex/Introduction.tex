\chapter*{Introduction}
\addcontentsline{toc}{chapter}{Introduction}
Lorsqu'une personne malvoyante compte se rendre à une destination dans une ville, des applications GPS lui sont disponibles grâce à une option de commande vocale, qui lui indique la route à suivre. Le problème que l'on peut constater avec cette voix qui lui sert de guide est que, pour l'entendre en pleine rue, l'utilisateur se voit obligé de brancher des écouteurs à l'appareil s'il ne veut pas mettre de haut-parleurs pour rester discret. Il serait donc utile à ces personnes, dont les oreilles sont pour eux le principal organe qui leur permettent de se situer dans l'espace, d'avoir une fonctionnalité qui leur permettrait une plus grande discrétion.\\

Le projet aura donc pour but de fournir à l'utilisateur une fonction de guidage par la vibration des téléphones, en donnant à l'utilisateur la possibilité de lier deux téléphones par une connexion \textit{BlueTooth} afin que le téléphone principal soit dédié à vibrer quand l'utilisateur doit tourner à gauche, et l'autre à droite, ou inversement.\\

Il est important de noter que le projet a pour but principal de nous apprendre à travailler en groupe sur un projet assez conséquent. Afin de gérer ce type de projet, une grande rigueur dans l'organisation est de mise.\\

La première étape était d'abord d'apprivoiser le fonctionnement de l'environnement mobile, de comprendre les concepts de la programmation sur système \textit{Android} et de mettre en place le dépôt SVN du projet sur le serveur \textit{Savane} du \textit{CREMI} afin d'assurer cette rigueur obligatoire pour les projets de cette échelle.\\

Dans une première partie, nous présenterons le contexte de développement et le cahier des charges du projet, qui comprend l'analyse des besoins, des schémas de scénarios et des diagrammes UML.
Dans un deuxième temps, nous détaillerons l'architecture du projet, le lien entre les différents modules du projet et leur évolution.
Ensuite, nous entamerons la question des tests, qu'ils soient unitaires ou d'intégration, et nous expliquerons le fonctionnement précis de l'application. Enfin, nous conclurons en détaillant à quel point en est le projet, les améliorations que l'on peut lui apporter et nous dresserons un bilan du cycle de développement que nous avons parcouru.